\chapter*{Résumé}
\pagenumbering{roman}
\thispagestyle{plain}
\addcontentsline{toc}{chapter}{Résumé}

Ce mémoire d'ingénieur explore la problématique de la conception d'un modèle d'intéressement financier au sein d'écosystèmes collaboratifs de partage de données. En mettant l'accent sur la valeur de la connaissance, le calcul des contributions et la faisabilité de l'implémentation sur une blockchain, cette étude vise à apporter des éclaircissements essentiels sur la manière d'allouer les ressources dans ces écosystèmes.

Cette étude adopte une approche novatrice en utilisant l'entropie comme métrique pour évaluer la valeur de la connaissance. Les résultats obtenus révèlent que l'entropie offre un moyen de quantifier la diversité et la pertinence des données, tout en mettant en évidence l'importance de la méthode de discrétisation pour obtenir des valeurs précises.

Ensuite, une solution de tarification de la connaissance est proposée à travers l'utilisation d'une fonction de paiement de Myerson, qui combine l'entropie avec une fonction d'altération pour servir de fonction d'allocation. Le résultat de cette approche est un système d'enchères entre les consommateurs successifs de connaissances.

En se penchant sur le calcul des contributions, il apparaît que la valeur de Shapley peut présenter des limitations dans les environnements où la fonction de valeur n'est pas superaditive. Cette observation remet en question son application pour une répartition équitable des ressources et suggère la nécessité d'explorer des alternatives plus adaptées.

En ce qui concerne la faisabilité d'une implémentation sur une blockchain, cette étude soulève des préoccupations quant à la complexité computationnelle de la solution. Bien que la blockchain puisse sembler une solution attrayante pour automatiser le processus avec ses garanties de transparence et de sécurité, les exigences en termes de puissance de calcul pourraient poser des défis majeurs.

Ce mémoire offre une contribution significative en décomposant les défis liés à l'évaluation et à la tarification des connaissances dans des environnements collaboratifs. Les résultats mettent en lumière des approches prometteuses pour quantifier la valeur de la connaissance et questionnent la pertinence de certaines méthodes de calcul des contributions. En outre, des réflexions sur l'implémentation pratique sur une blockchain offrent des perspectives sur d'autres formes d'implémentations.

\textbf{Mots clés :} Tarification des données, valeur de Shapley, partage de données, blockchain, économie de la connaissance

\textbf{Référence bibliographique du mémoire :} YEBOUET, Cédrick-Armel, 2023. Conception d'un modèle d'intéressement financier pour écosystèmes de partage de données décentralisés. Mémoire d’Ingénieur Agronome, option AgroTIC, L'institut Agro Montpellier.

\newpage

\chapter*{Abstract}
\thispagestyle{plain}
\addcontentsline{toc}{chapter}{Abstract}

This engineering thesis explores the problem of designing a financial profit-sharing model within collaborative data-sharing ecosystems. By focusing on the value of knowledge, the calculation of contributions and the feasibility of implementation on a blockchain, this study aims to provide essential insights into how to allocate resources in these ecosystems.

This study takes an innovative approach by using entropy as a metric to assess the value of knowledge. The results reveal that entropy offers a means of quantifying the diversity and relevance of data, while highlighting the importance of the discretization method in obtaining accurate values.

Next, a knowledge pricing solution is proposed through the use of a Myerson payment function, which combines entropy with a taint function to serve as an allocation function. The result of this approach is an auction system between successive consumers of knowledge.

Looking at the calculation of contributions, it appears that Shapley's value may present limitations in environments where the value function is not superaditive. This observation calls into question its application for equitable resource allocation and suggests the need to explore more suitable alternatives.

As for the feasibility of implementation on a blockchain, this study raises concerns about the computational complexity of the solution. While blockchain may seem an attractive solution for automating the process with its guarantees of transparency and security, the requirements in terms of computing power could pose major challenges.

This dissertation offers a significant contribution by breaking down the challenges associated with knowledge valuation and pricing in collaborative environments. The results highlight promising approaches to quantifying the value of knowledge, and question the relevance of certain methods for calculating contributions. In addition, reflections on practical implementation on a blockchain offer perspectives on other forms of implementation.

\textbf{Keywords :} Data pricing, Shapley value, Data sharing, Blockchain, knowledge economy

\textbf{Citation :} YEBOUET, Cédrick-Armel, 2023. Conception d'un modèle d'intéressement financier pour écosystèmes de partage de données décentralisés. Mémoire d’Ingénieur Agronome, option AgroTIC, L'institut Agro Montpellier.

\newpage

