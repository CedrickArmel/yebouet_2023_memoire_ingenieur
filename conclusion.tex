\chapter*{CONCLUSION}
\thispagestyle{plain}
\addcontentsline{toc}{chapter}{Conclusion}

En conclusion, ce mémoire d'ingénieur a plongé dans les profondeurs de la valorisation des données au sein des écosystèmes collaboratifs de partage, en mettant en lumière des défis fondamentaux et en proposant des pistes prometteuses pour les résoudre. L'évolution vers une ère numérique a imposé la reconnaissance de la valeur intrinsèque des données, non seulement pour l'innovation et les stratégies industrielles, mais aussi pour la création de connaissances et la réalisation de leur plein potentiel.

L'agriculture, en tant que secteur clé, a été au cœur de cette révolution, marquée par l'abondance de données générées à chaque étape de la chaîne de valeur. Cependant, l'éparpillement de ces données au sein des entreprises et institutions a entravé leur exploitation optimale. Les barrières au partage étaient nombreuses, allant des contraintes financières à la sécurité des données, et de l'absence de normes à la rivalité commerciale. L'émergence d'OKP4 a apporté une réponse innovante en proposant des écosystèmes collaboratifs soutenus par la blockchain, déverrouillant ainsi le potentiel inexploité des données partagées.

Ce mémoire a approfondi la problématique en se concentrant sur la conception d'un modèle d'intéressement financier au sein de ces écosystèmes. L'exploration de l'utilisation de l'entropie pour évaluer la valeur des connaissances a ouvert de nouvelles perspectives pour quantifier la pertinence et la diversité des données. De plus, la proposition d'une fonction de paiement de Myerson basée sur l'entropie a ouvert la voie à des systèmes d'enchères entre les consommateurs successifs de connaissances.

Le calcul des contributions, bien que complexe, a été abordé à travers la valeur de Shapley, révélant ses limites dans des environnements non superadditifs et suggérant la nécessité de rechercher des alternatives plus adaptées.

Enfin, l'implémentation sur une blockchain, bien que porteuse de transparence et de sécurité, soulève des préoccupations quant à la complexité computationnelle.

\newpage
\thispagestyle{empty}
\null
\newpage