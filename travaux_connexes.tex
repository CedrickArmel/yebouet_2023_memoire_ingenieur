\subsection{Travaux connexes}\label{subsec:travaux_connexes}

Cette étude, qui s'intéresse à la valorisation des connaissances dans l'écosystème OKP4, s'inscrit dans le champ plus large de la tarification des données, un domaine en pleine expansion. La complexité de la question traitée en matière de tarification des données positionne ce domaine à l'intersection de plusieurs disciplines, notamment la théorie de l'information, la théorie des enchères et la théorie des jeux, comme c'est le cas dans cette étude.

La théorie de l'information a été développée initialement par Claude Shannon en 1948. Elle est principalement concernée par la quantification de l'information. Cette théorie est utilisée pour comprendre comment l'information peut être encodée et transmise de manière efficace. Dans le contexte de la tarification des données, comprendre la "valeur de l'information" peut être crucial. Par exemple, certaines données peuvent être plus "informatives" que d'autres, et donc plus précieuses.

La théorie des enchères étudie comment les biens et les services sont alloués par le mécanisme des enchères. Cette théorie cherche à comprendre les stratégies optimales pour les acheteurs et les vendeurs sous différentes règles d'enchère. Les théoriciens des enchères, tels que William Vickrey, ont développé des modèles pour différents types d'enchères comme les enchères anglaises, hollandaises, et de Vickrey. Dans le domaine de la tarification des données, la théorie des enchères peut être utilisée pour déterminer comment les données peuvent être vendues au prix le plus élevé possible tout en étant équitable pour tous les participants.

La théorie des jeux est une branche des mathématiques qui étudie les interactions stratégiques entre différents "joueurs" ou agents dans un environnement donné. La théorie des jeux peut être coopérative ou non coopérative, et elle a été utilisée dans une multitude de domaines allant de l'économie à la biologie. Dans le contexte de la tarification des données, la théorie des jeux peut aider à comprendre comment différents agents (comme les acheteurs et les vendeurs de données) interagissent stratégiquement. Des concepts comme la "valeur de Shapley" peuvent être utilisés pour diviser équitablement la valeur générée par une coalition de fournisseurs de données.

