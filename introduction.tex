\vspace*{-1cm}
\chapter*{INTRODUCTION}
\pagenumbering{arabic}
\setcounter{page}{1}
\addcontentsline{toc}{chapter}{Introduction}

À l'aube de l'ère numérique, les données sont devenues une ressource précieuse. Leur capacité à catalyser l'innovation et à façonner les stratégies industrielles est devenue incontestable. Un secteur qui s'est particulièrement illustré dans cette quête de données est l'agriculture, grâce aux avancées technologiques qui ont ouvert les vannes d'une abondance de données auparavant inimaginable. Cependant, au cœur de cette abondance, se trouve un paradoxe majeur : l'insuffisance de leur agrégation et de leur exploitation efficace.

Le secteur agricole génère une multitude de données à chaque étape, de la croissance des cultures à la commercialisation des produits. Cependant, ces données restent souvent cloisonnées au sein des entreprises et des institutions, limitant ainsi leur potentiel de création de valeur. Les barrières au partage de données sont multiples : manque d’incitations financières adaptées, préoccupations de sécurité, absence de normes, problèmes de confidentialité et rivalités commerciales.

C'est dans ce contexte qu'émerge OKP4, une solution technologique révolutionnaire qui propose de déverrouiller les portes du partage de données à travers des écosystèmes collaboratifs, renforcés par la puissance de la blockchain. En offrant un cadre technologique permettant la création de \textit{data spaces}, OKP4 transcende les cloisonnements traditionnels, avec un environnement sécurisé et transparent où les acteurs économiques peuvent partager leurs données pour un bénéfice mutuel. Cette approche s'inscrit dans la vision de l'économie de la connaissance, où la création et la diffusion de connaissances sont valorisées et récompensées.

Dans le cadrage économique de sa méthode pour la construction d'écosystèmes de partage de données viables en agriculture, \citeauthor{pelliet_pourquoi_2021} (\citeyear{pelliet_pourquoi_2021}) énonce un ensemble de questions alors restées sans réponses notamment : comment mésurer la valeur d'une connaissance ? quel mécanisme pour la fixation du prix d'une connaissance ? comment redistribuer la valeur aux contributeurs de l'écosystème de partage ?

C'est dans cette optique que le présent mémoire examine divers éléments de réponse à ces questions en se focalisant sur la conception d'un modèle d'intéressement financier. De plus, la faisabilité de l'exécution de ce modèle sur une blockchain est discutée. Par l'analyse des multiples facettes de cette problématique, l'objectif est de contribuer de manière significative à la compréhension de la valorisation des données au sein des écosystèmes de partage de données.

A cette fin, ce mémoire suit une progression en quatre chapitres. Le premier chapitre présente des éléments de contexte sur l'usage de la donnée, les barrières et l'apport subsentiel de l'entreprise OKP4 dans l'émergence d'écosystèmes de partage en agriculture. Le deuxième chapitre plante le décor technologique de ce mémoire à travers une présentation de l'écosystème OKP4. Au chapitre trois, la problématique d'un modèle d'intéressement financier est examinée en détail, avant d'aborder une méthodologie susceptible d'apporter des éléments de réponse. Enfin, dans le dernier chapitre, les résultats sont présentés et discutés.


